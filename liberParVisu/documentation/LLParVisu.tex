\documentclass{article}
%\setlength{\textwidth}{15cm}

%\documentclass[submit]{aiaa-tc}% insert '[draft]' option to show overfull boxes

\usepackage[margin=1.3in]{geometry}
 \usepackage{wrapfig}% embedding figures/tables in text (i.e., Galileo style)
 \usepackage{threeparttable}% tables with footnotes
 \usepackage{dcolumn}% decimal-aligned tabular math columns
  \newcolumntype{d}{D{.}{.}{-1}}
 \usepackage{nomencl}% automatic nomenclature generation via makeindex
  \makeglossary
 \usepackage{subfigure}% subcaptions for subfigures
 \usepackage{subfigmat}% matrices of similar subfigures, aka small mulitples
 \usepackage{fancyvrb}% extended verbatim environments
  \fvset{fontsize=\footnotesize,xleftmargin=2em}
 \usepackage{lettrine}% dropped capital at beginning of paragraph
 \usepackage[dvips]{dropping}% alternative dropped capital package
 \usepackage{hyperref}% embedding hyperlinks [must be loaded after dropping]
 \usepackage{amsmath}
 \usepackage{color}
 \usepackage{setspace}
 \usepackage{epsf}
 \usepackage{epsfig}
% \renewcommand{\baselinestretch}{1.25}

\begin{document}

\begin{titlepage}
\begin{center}

% Upper part of the page. The '~' is needed because \\
% only works if a paragraph has started.
\includegraphics[width=0.25\textwidth]{figures/LLlogo.eps}~\\[1cm]

\textsc{\LARGE LiberLocus}\\[1.5cm]

% Title
{ \huge \bfseries LLParVisu Library}\\[0.4cm]

\textsc{\Large LiberLocus scientific output generation library for parallel software}\\[0.5cm]

\textsc{Version 1.4}\\[3.5cm]

\textsc{\Large LiberLocus Team}\\[1.5cm]

\textsc{\Large Earth}\\[0.5cm]

\vfill

% Bottom of the page
{\large \today}

\end{center}
\end{titlepage}

%\maketitle
%\section{Introduction}

\section{Introduction}
\label{sec:intro}

This is a documentation about the LiberLocus LLParVisu library. LLParVisu library is a library built upon LLVisu library. LLParVisu aims to bring parallel computing capability to the already existing capabilities in LLVisu.

For detailed description regarding serial part of the LLParvisu please refer to the LLVisu manual. This document consists of only the parallel capabilities built upon LLVisu.

The supported file formats are VTK\cite{VTK} and XDMF\cite{XDMF} (through HDF5\c
ite{HDF5}). User can generate output data with one or multiple types simultaneou
sly with the appropriate function calls.

The installation procedure is completed with CMake\cite{CMake}. LLVisu must already be installed on the system as LLParVisu is built on top of LlVisu. Once installed, by including the appropriate header file in your code, functions can be called for output generation. Detailed explanation of these functions can be found in Section~\ref{sec:API}

%\section{What is new in Version 2.0}
%\label{sec:whatsnew}

\newpage

\section{LLParVisu API}
\label{sec:API}

\noindent {\large void {\bf liberParVisuVTK}(\newline \par {\it string}  fileName, {\it int}  nodeNum, {\it float}  *pos\_x, {\it float}  *pos\_y, {\it float}  *pos\_z, {\it int}  cellNum, {\it int}  nodePerCell, {\it int}  **cells, {\it char}  **varName, {\it int}  **varSize, {\it float}  **varMatrix, {\it int}  mpiRank \newline),  }
\bigskip

\noindent {\large {\bf Description}}
        \par This function creates a vtk type output file in vtu format.
\bigskip
        
\noindent {\large {\bf Parameters}}
        \par Below are the definitions of the parameters in the function.
\medskip

        {\bf fileName}  \par the absolute path to the output directory + filename
\medskip
        
        {\bf nodeNum} \par number of nodes present in the mesh
\medskip
        
        {\bf pos\_x} \par pointer to the array that stores the x position of the nodes
\medskip
        
        {\bf pos\_y} \par pointer to the array that stores the y position of the nodes
\medskip
        
        {\bf pos\_z} \par pointer to the array that stores the z position of the nodes
\medskip
        
        {\bf cellNum} \par number of cells present in the mesh
\medskip
        
        {\bf nodePerCell} \par number of nodes per cells 
\medskip
        
        {\bf cells} \par pointer to the two dimensional array that stores the cell connectivity
\medskip
        
        {\bf varName} \par pointer to the array that stores the variable names
\medskip
        
        {\bf varSize} \par number of variables that are passed 
\medskip
        
        {\bf varMatrix} \par pointer to the two dimensional array that stores the variable value arrays(cell or node based) 
\medskip
        
        {\bf mpiRank} \par processor id

\newpage

\noindent {\large void {\bf liberParVisuXMF}(\newline \par {\it string}  fileName, {\it string}  gridName, {\it string}  topoType, {\it int}  nodeNum, {\it float}  *pos\_x, {\it float}  *pos\_y, {\it float}  *pos\_z, {\it int}  cellNum, {\it int}  nodePerCell, {\it int}  **cells, {\it char}  **varName, {\it char}  **varType, {\it int}  *varLength, {\it int}  varSize, {\it float}  **varMatrix, {\it int}  mpiRank \newline) }
\bigskip

\noindent {\large {\bf Description}}
        \par This function creates a xdmf type output file in xmf format using hdf5.
\bigskip
        
\noindent {\large {\bf Parameters}}
        \par Below are the definitions of the parameters in the function.
\medskip

        {\bf fileName}  \par the absolute path to the output directory + filename
\medskip
        
        {\bf nodeNum} \par number of nodes present in the mesh
\medskip
        
        {\bf pos\_x} \par pointer to the array that stores the x position of the nodes
\medskip
        
        {\bf pos\_y} \par pointer to the array that stores the y position of the nodes
\medskip
        
        {\bf pos\_z} \par pointer to the array that stores the z position of the nodes
\medskip
        
        {\bf cellNum} \par number of cells present in the mesh
\medskip
        
        {\bf nodePerCell} \par number of nodes per cells 
\medskip
        
        {\bf cells} \par pointer to the two dimensional array that stores the cell connectivity
\medskip
        
        {\bf varName} \par pointer to the array that stores the variable names
\medskip
        
        {\bf varSize} \par number of variables that are passed 
\medskip
        
        {\bf varMatrix} \par pointer to the two dimensional array that stores the variable value arrays(cell or node based) 
\medskip
        
        {\bf mpiRank} \par processor id 

  \begin{thebibliography}{1}
  
  \bibitem{VTK} Visualization Tool Kit, Page {\em http://www.vtk.org/}
  \bibitem{XDMF} Extensible Data Model Format, Page {\em http://www.xdmf.org}
  \bibitem{HDF5} Hierarchical Data Format Group, Page {\em http://www.hdfgroup.org/HDF5/}
  \bibitem{CMake} Cross-platform, open-source build system, Page {\em http://www.cmake.org/}
%
%  \bibitem{wiki1} Sorting Algorithm Wikipedia, Page {\em http://en.wikipedia.org/wiki/Sorting\_algorithm}
%  \bibitem{wiki2} Mergesort Wikipedia Page, {\em http://en.wikipedia.org/wiki/Merge\_sort}
%  \bibitem{main} Atanas Radenski "{\em Shared Memory, Message Passing, and Hybrid Merge Sorts for Standalone and Clustered SMPs}"  International Conference on Parallel and Distributed Processing Techniques and Applications, 2011 pp. 367-373.
%  \bibitem{sec} Jaeyoung Park, Kyong-Gun Lee and Jong Tae Kim "{\em Parallel Mergesort Implementation using OpenMP}" Proceedings of The 2011 World Congress in Computer Science, Computer Engineering and Applied Computing. "http://world-comp.org/p2011/PDP8400.pdf"
%
  \end{thebibliography}

\end{document}
